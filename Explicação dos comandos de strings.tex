\documentclass[12pt]{article}
\usepackage[utf8]{inputenc}
\usepackage{amsmath}
\usepackage{listings}
\usepackage{xcolor}
\usepackage{abntex2cite} % Formato ABNT para citações
\usepackage{geometry} 
\geometry{left=2.5cm,right=2.5cm,top=2.5cm,bottom=2.5cm}

% Estilo do código fonte
\lstset{
    language=C,
    basicstyle=\ttfamily\small,
    keywordstyle=\color{blue},
    commentstyle=\color{gray},
    stringstyle=\color{green!60!black},
    showstringspaces=false,
    breaklines=true,
    backgroundcolor=\color{lightgray!20}, % Fundo do código
    frame=single, % Moldura ao redor do código
    numbers=left, % Adiciona número nas linhas
    numberstyle=\tiny\color{gray}
}

% Estilo da capa
\title{
    {\normalsize Instituto Federal do Espírito Santo - IFES} \\
    {\normalsize Campus Cariacica} \\
    \vspace{2cm}
    \textbf{Relatório sobre Comandos da Biblioteca \texttt{string.h} em C}
    \vspace{2cm}
}
\author{André Cunha - Enpro 4}
\date{\today}

\begin{document}

% Capa
\maketitle
\newpage

% Sumário
\tableofcontents
\newpage

\section{Introdução}
No contexto da linguagem C, a biblioteca \texttt{string.h} fornece um conjunto de funções para a manipulação de strings, que são arrays de caracteres terminados pelo caractere nulo (\texttt{'\textbackslash 0'}). A compreensão das principais funções dessa biblioteca é fundamental para o desenvolvimento de programas que lidam com dados textuais.

Neste relatório, exploramos algumas das funções mais comuns da biblioteca, ilustrando seus usos com exemplos práticos e destacando sua importância no desenvolvimento de software em C.

\section{Comandos Principais}

A seguir, descreveremos as principais funções da biblioteca \texttt{string.h}, acompanhadas de exemplos práticos para cada uma delas.

\subsection{\texttt{strlen()}: Calculando o Tamanho de uma String}
A função \texttt{strlen()} retorna o tamanho de uma string, ou seja, a quantidade de caracteres até o caractere nulo (\texttt{'\textbackslash 0'}). Ela é amplamente utilizada para determinar o comprimento de strings armazenadas em arrays de caracteres.

\textbf{Assinatura da função:}

\begin{lstlisting}
size_t strlen(const char *str);
\end{lstlisting}

\textbf{Exemplo Prático:}

Imagine que você tenha uma string que representa uma frase ou palavra. Vamos medir seu comprimento:

\begin{lstlisting}
char str[] = "C Programming";
printf("Tamanho da string: %lu\n", strlen(str));
\end{lstlisting}

No exemplo acima, a string \texttt{"C Programming"} tem 13 caracteres, e a função \texttt{strlen()} exibirá esse valor.

\subsection{\texttt{strcpy()}: Copiando Strings}
A função \texttt{strcpy()} copia o conteúdo de uma string para outra. Isso é útil quando queremos duplicar o conteúdo de uma string para realizar modificações sem alterar o original.

\textbf{Assinatura da função:}

\begin{lstlisting}
char *strcpy(char *dest, const char *src);
\end{lstlisting}

\textbf{Exemplo Prático:}

Pense no caso em que precisamos copiar uma palavra de uma variável para outra:

\begin{lstlisting}
char src[] = "Hello";
char dest[10];
strcpy(dest, src);
printf("Destino: %s\n", dest);
\end{lstlisting}

Aqui, a string \texttt{"Hello"} é copiada para a variável \texttt{dest}, e o programa exibirá o valor \texttt{"Hello"} na saída.

\subsection{\texttt{strcat()}: Concatenando Strings}
A função \texttt{strcat()} é usada para concatenar (adicionar) uma string ao final de outra. Essa função é particularmente útil quando precisamos unir duas strings para formar uma frase completa.

\textbf{Assinatura da função:}

\begin{lstlisting}
char *strcat(char *dest, const char *src);
\end{lstlisting}

\textbf{Exemplo Prático:}

Vamos criar uma saudação completa ao concatenar duas palavras:

\begin{lstlisting}
char str1[20] = "Hello";
char str2[] = " World";
strcat(str1, str2);
printf("%s\n", str1); // Resultado: Hello World
\end{lstlisting}

O conteúdo de \texttt{str1} se transforma em \texttt{"Hello World"} após a concatenação com \texttt{str2}.

\subsection{\texttt{strcmp()}: Comparando Strings}
A função \texttt{strcmp()} compara duas strings lexicograficamente, ou seja, de acordo com a ordem alfabética. É uma função fundamental para verificar se duas strings são iguais ou se uma vem antes ou depois da outra.

\textbf{Assinatura da função:}

\begin{lstlisting}
int strcmp(const char *str1, const char *str2);
\end{lstlisting}

\textbf{Exemplo Prático:}

Imagine que estamos comparando duas palavras para saber qual delas viria primeiro em um dicionário:

\begin{lstlisting}
char str1[] = "apple";
char str2[] = "banana";
int result = strcmp(str1, str2);
if (result < 0) {
    printf("str1 vem antes de str2.\n");
} else if (result > 0) {
    printf("str1 vem depois de str2.\n");
} else {
    printf("As strings são iguais.\n");
}
\end{lstlisting}

Se \texttt{str1} for menor que \texttt{str2}, o valor retornado será negativo; caso contrário, será positivo.

\subsection{\texttt{strstr()}: Buscando Substrings}
A função \texttt{strstr()} busca uma substring dentro de outra string, retornando um ponteiro para a primeira ocorrência encontrada. Se a substring não for encontrada, ela retorna \texttt{NULL}.

\textbf{Assinatura da função:}

\begin{lstlisting}
char *strstr(const char *haystack, const char *needle);
\end{lstlisting}

\textbf{Exemplo Prático:}

Vamos verificar se a palavra "gram" aparece em "C Programming":

\begin{lstlisting}
char str[] = "C Programming";
char *sub = strstr(str, "gram");
if (sub != NULL) {
    printf("Substring encontrada: %s\n", sub);
} else {
    printf("Substring não encontrada.\n");
}
\end{lstlisting}

Neste exemplo, a função \texttt{strstr()} localiza a substring \texttt{"gram"} dentro da string principal e exibe o resultado a partir do ponto onde foi encontrada.

\section{Conclusão}
As funções da biblioteca \texttt{string.h} desempenham um papel crucial no desenvolvimento de software em C que lida com dados textuais. A manipulação eficiente de strings permite a criação de programas robustos e flexíveis, capazes de processar dados de maneira precisa e eficiente. Neste relatório, exploramos as funções mais importantes da biblioteca, ilustrando seu uso com exemplos claros e aplicáveis.

\end{document}
